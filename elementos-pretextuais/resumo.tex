Uma máquina de vetores suporte (MVS) é um método de aprendizado de máquina.
Usamos um conjunto de dados para ensinar a máquina a distinguir entre dois grupos conhecidos. Depois de treinada, a máquina é capaz de prever a qual grupo pertencem novas entradas que não foram vistas antes.
A etapa de treinamento consiste em iterar diversas vezes sobre todo o conjunto de treinamento fazendo cálculos com os dados. 
Na etapa de classificação fazemos cálculos com a nova entrada, usando valores obtidos no treinamento, para prever a qual classe pertence essa entrada.
Esse processo é muito custoso, já que o conjunto de dados precisa ser grande para ter uma boa porcentagem de acertos na classificação, por isso é um método interessante para ser paralelizado em GPU. GPUs possuem uma capacidade de processamento paralelo muito acima das CPUs, mas o seu uso não é trivial e é preciso adaptar os algoritmos à sua arquitetura. 
Neste trabalho implementamos um programa capaz de executar uma máquina de vetores suporte de forma sequencial e paralela. O algoritmo escolhido para implementá-la foi Kernel-Adatron 
por ser um algoritmo simples de MVS que tem um potencial de ganho com a paralelização em GPU.
Avaliamos o algoritmo em relação à precisão obtida na classificação e ao ganho de desempenho da versão paralela em relação à versão sequencial.
